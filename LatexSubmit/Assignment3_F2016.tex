\documentclass{assignment}

\coursetitle{Introduction to Cryptography}
\courselabel{CPSC 418}
\exercisesheet{Home Work \#3}{}
\student{Mike Simister - 10095107}
\semester{Fall 2016}
%\usepackage[pdftex]{graphicx}
%\usepackage{subfigure}
\usepackage{amsfonts}
\usepackage[fleqn]{amsmath}
\usepackage[normalem]{ulem}
\usepackage{amsthm}
\usepackage{physics}
\usepackage[singlespacing]{setspace}

\begin{document}
\sloppy

\begin{center}
\renewcommand{\arraystretch}{2}
\begin{tabular}{|c|c|c|} \hline
Problem & Marks \\ \hline \hline
1 & \\ \hline
2 & \\ \hline
3 & \\ \hline
4 & \\ \hline
5 & \\ \hline
6 & \\ \hline
7 & \\ \hline \hline
Total & \\ \hline
\end{tabular}
\end{center}
\newpage

\begin{flushleft}
\begin{problemlist}
\pbitem (Flawed hash function based MAC designs, 28 marks)
%\begin{problem}
%{answer}
\item[(a)] \hspace{1cm}\\
\begin{enumerate}
\item[(i)]\hspace{1cm}\\
\begin{align*}
M_1 &= f(f(IV,K),M_1)\\
&= f(H_1,M'_1) \leftarrow \text{ note: } M_1 \text{ is now padded } \\
&= H(M)\\
&= H_L\\
&= MAC_1\\
\hspace{1cm}\\
M_2 &= H(K||M'_1||X)\\
&= f(f(f(IV,K),M'_1),X))\\
&= f(MAC_{1},X)\\
&= MAC_{2}\\
\end{align*}
\item[(ii)]\hspace{1cm}\\
\begin{align*}
C_k(M) &= H(M'||K) \text{ where M' is M with padding}\\
M_1 &= f(IC,M)\\
&= f(f(IV,M'_1),K)\\
&= H(M)\\
&= H_L\\
&= MAC_1\\
\hspace{1cm}\\
M_2 &= f(IV,M_2) \\
&= f(f(IV,M),K) \leftarrow \text{ since } H(M_2) = H(M_1) \\
&= H(M)\\
&= HL\\
&= MAC_1\\  
\end{align*}
\end{enumerate}
\newpage
\item[(b)]\hspace{1cm}\\
\begin{enumerate}
\item[(i)]\hspace{1cm}\\
\begin{doublespace}
$M_1 \text{ is one block so no padding, } MAC_1 = E_k(M_1)$ \\
$M_2 \text{ is one block so no padding, } MAC_2 = E_k(M_2)$ \\
$M_3 \text{ is two blocks i.e. } \begin{cases} \text{Block 1 } = M_1\\
\text{Block 2 }= \left\{0\right\}^n\\\end{cases}$\\
\hspace{1cm}\\
$MAC_3 = E_k(M_1 \oplus \left\{0\right\}^n)$\\
\hspace{1.1cm}$      = E_k(MAC_1)$\\
\item[(ii)]
$M_1 \text{ is one block so no padding, } MAC_1 = E_k(M_1)$ \\
$M_2 \text{ is one block so no padding, } MAC_2 = E_k(M_2) $\\
$M_3 \text{ is two blocks so it will take 2 iterations to compute } MAC_3$\\
$\text{Iteration 1  }E_k(M_1) = MAC_1$\\
$\text{Iteration 2 } E_k(MAC_1 \oplus X) = MAC_3$\\
\hspace{1cm}\\
$M_4 \text{ is two blocks so it will take 2 iterations to compute } MAC_4$\\
$\text{Iteration 1 }E_k(M_2) = MAC_2$\\
$\text{Iteration 2 } E_k(MAC_1 \oplus X \oplus MAC_2 \oplus MAC_2)$\\
$\text{Iteration 2 } E_k(MAC_1 \oplus X) = MAC_3 \leftarrow$ Since $MAC_2 \oplus MAC_2$ cancel\\
\end{doublespace}
\end{enumerate}
%\end{answer}
%\end{problem}
\newpage
%\begin{problem}
%\begin{answer}
\pbitem (A modified man-in-the-middle attack on Diffie-Hellman, 12 marks)
\item[(a)] \hspace{1cm}\\
\begin{tabular}{c | c | c}\\
Alice &Mallory &Bob\\ \hline
chooses $a$ &chooses $q$		&chooses $b$\\
computes $g^a$ &	&computes $g^b$\\
sends $g^a \rightarrow$ &intercepts  $g^a$ sends $g^{aq} \rightarrow$ &recieves $g^{aq}$\\
recieves $g^{bq} \leftarrow$ &$\leftarrow g^{bq}$ sends  intercepts  $g^b$   &$\leftarrow g^{b}$ sends \\ 
computes $K = g^{abq}$	& 	&computes $K = g^{abq}$\\ 
\end{tabular}
\item [(b)]\hspace{1cm}\\
Things we know:
\begin{enumerate}
\item[(1)] g is a primitive root, or a generator in $\mathbb{Z}^{*}_{p}$\\
\item[(2)] the order of $g$ is $|\mathbb{Z}^{*}_{p}| = (p-1)$\\
\item[(3)] $p = mq + 1$, $(p-1) = mq$\\
\item[(4)] $g^{(p-1)} \equiv 1 \pmod{p}$, $g^{mq} \equiv 1 \pmod{p}$\\
\item[(5)] $g^{2mq} \pmod{p} \equiv g^{mq} * g^{mq} \pmod{p} \equiv 1 * 1 \pmod{p} \equiv 1 \pmod{p}$\\
\item[(6)] $g^{(k)mq} \pmod{p} \equiv 1 \pmod{p} $ where $k \in \mathbb{Z}_{0\geq}$, that is $0 \leq k$\\
\end{enumerate}
\begin{doublespace}
So, if $ab \geq m$, we can say that $abq = ((k)m + r)q$ where $0 \leq r < m$, $1 \leq k$\\
and $g^{(k)mq} \equiv 1 \pmod{p}$\\
So, when $ab = (k)m$, $g^{abq} \equiv g^{(k)mq} \equiv 1 \pmod{p}$\\
Otherwise, $g^{abq} = g^{((k)m + r)q} \equiv g^{(k)mq} * g^{rq} \equiv 1 * g^{rq} \pmod{p}$\\
And since, $g$ is a generator, and q is a constant the values obtained from calculating $g^{0}\pmod{p}$ to $g^{qr} \pmod{p}$ are unique values
and since $0 \leq r < m$ there are $m$ possible values\\
\end{doublespace}
\item[(c)]\hspace{1cm}\\
The advantage of this variation is that once the attacker, Mallory, has intercepted\\
and modified both $g^{aq}$ and $g^{bq}$ there is no need to do any further work.\\
That is to say, Alice and Bob can continue communicating completely unaware that\\
Mallory has tampered with their communication. Mallory can easily compute the key\\
and decrypt/encrypt any/all messages she chooses. In the version of discussed in\\
class, after Mallory intercepts $g^{a}$ and $g^{b}$ sends $g^{ae}$ or $g^{be}$ Mallory would\\
need to continuously intercept and encrypt/decrypt ALL messages. Otherwise, if\\
even one message is not intercepted by Mallory, the decryption by the intended\\
recipient will fail, and alert the recipient that the system has been compromised.\\
%\end{answer}
%\end{problem}
\newpage
\pbitem (Binary Exponentiaion, 12 marks)
\item [(a)] Define $s_0 = b_0$ and $s_{i+1} = 2_{Si} + b_{i+1}$\\
\begin{flushleft}
\begin{align*}
\text{Suppose that: } & S_i = \sum_{n=0}^{i} b_n2^{i-n}\\  
\text{We want to show that: }  &S_{i+1} = \sum_{n=0}^{i+1} b_n2^{i+1-n}\\ 
\text{Base Case: } &i = 0\text{,   }\\
&S_{0} = \sum_{n=0}^{0} b_n2^{0} = b_02^0 = b_0\\ 
\text{So, } s_{i+1} &= 2_{Si} + b_{i+1}\\
&= 2\left(\sum_{n=0}^{i} b_n2^{i-n}\right) + b_{i+1}\\
&= 2\left(b_0s^{i-0}\right) + 2\left(b_i2^{i-1}\right) + . . . + 2\left(b_i2^{i-i}\right) + b_{i+1}\\
&= b_02^{i+1} + b_12{i} + . . . + b_i2^1 + b(i+1)(2^0)\\
S_{i+1} &= \sum_{n=0}^{i+1} b_n2^{i+1-n} \qed\\ 
\end{align*}
\end{flushleft}
\item[(b)] 
\begin{align*}
\text{Suppose } r_i \equiv a^{S_i} \pmod{m}\\
\hspace{1cm}\\
\text{Base Case } r_0 &\equiv a^1 \pmod{m}\\
r_0 &\equiv a \pmod{m}\text{  base case proved} \text{ }\checkmark\\
\hspace{1cm}\\
\text{W.T.S. } r_{i+1} \equiv a^{S_{i+1}} \pmod{m} \text{ for } 0 \leq i \leq K \\
\hspace{1cm}\\
\text{by the algorithm definition, we are given: }\\
\hspace{1cm}\\
r_{i+1} &= \begin{cases}r^{2}_i \pmod{m} \text{ if } b_{i+1} = 0,\\
r^{2}_ia \pmod{m} \text{ if } b_{i+1} = 1\end{cases}\\
\hspace{1cm}\\
\text{So, in case 1 where } b_{i+1} = 0\text{,}\\
\hspace{1cm}\\
r_{i+1} &= r^{2}_{i}\\
r_{i+1} &= r_i * r_i\\
&= a^{S_i} * a^{S_i} \textbf{ By the I.H.}\\
&=a^{2S_i}\\
\text{since } b_{i+1} = 0 \text{ we can write: } &a^{2S_i} \text{ as } a^{2S_{i} + b_{i+1}}\\
\text{and by defn in part (a) above: } &2S_i + b_{i+1} = S_{i+1}\\
\text{so, } a^{2S_i} &= a^{S_{i+1}} = r_{i+1}\\
\hspace{1cm}\\
\text{in case 2, where } b_{i+1} = 1\\
r_{i+1} &= r^{2}_{i} * a\\
r_{i+1} &= r_1 * r_i  * a\\
&= a^{S_i} * a^{S_i} * a \textbf{ by I.H. }\\
&= a^{2S_{i} + 1}\\
\text{ since } b_{i+1} &= 1\text{, }\\
a^{2S_{i} + 1} &= a^{2S{i} + b_{i+1}}\\
\text{and by the same defn in part (a), we can say that: }\\
a^{2S{i} + b_{i+1}} &= a^{S_{i+1}} = r_{i+1} \qed\\ 
\end{align*}
\newpage
\item [(c)]\hspace{1cm}\\
\begin{align*}
a_n &\equiv r_k \pmod{m} \text{, let } r_k = r_i\\
a_n &\equiv r_i \equiv a^{S_i} \pmod{m}\\
\text {where: }\\
n &= b_02^k + b_12^{k-1} + ... + b_{k-1}2 + b_0 \textbf{ by defn}\\
S_i &= b_02^{i-0} + b_i2^{i-1} + ... + b_i2^{i-i}\textbf{ by unwrapping the }\sum \textbf{ in part (a)}\\
\text{so, }\\
n &= S_i\\
\text{ so, }\\
a^n &\equiv a^{S_i} \equiv r_i \pmod{m}\qed\\
\end{align*}
\newpage
\pbitem (RSA toy example for practicing binary exponentiaion)
\item [(a)] \hspace{1cm}\\
Using the binary exponentiation algorithm where, $a = 17, n = 11, m = 77$\\
to calculate $a^n \pmod m$ since:\\
\begin{singlespace}
\begin{align*}
C &\equiv M^e \pmod n\\
&\equiv 17^{11} \pmod {77}\\
\end{align*}
so $n = 1011$\\
\begin{align*}
r_0 &= a \pmod n\\
&= 17 \pmod {77}\\
\hspace{1cm}\\
r_1 &= (r_{0})^2 \pmod {77}\\ 
&= 17^2 \pmod {77}\\ 
&= 58 \pmod {77}\\
\hspace{1cm}\\
r_2 &= (r_1)^2 * 17 \pmod {77}\\
&= 58^2 * 17 \pmod {77}\\
&= 54 \pmod {77}\\
\hspace{1cm}\\
r_3 &= (r_2)^2 * 17 \pmod {77}\\
&= 54^2 * 17 \pmod {77}\\
&= 61 \pmod {77}\\
\end{align*}
so, $C = 61$\\
\end{singlespace}
\newpage
\item [(b)] 
\begin{align*}
n &= pq \text{ where } p,q  \text{ are prime }\\
77 &= pq\\
&= 7 * 11\\
&\text{ so }\\
\phi(n) &= 6 * 10\\
&=60\\ 
&\text{ we want } d \text{ such that } de \equiv 1 \pmod {\phi(n)}\\
&\text{ or } (d)11 \equiv 1 \pmod{60}\\
\end{align*}
Extended Euclidean Algorithm:\\
\hspace{1cm}\\
GCD(60,11) = \\
\hspace{1cm}\\
\begin{tabular}{c c c c r l}
A &B &Q &R &\hspace{0.5cm} Factors &\hspace{1cm}\\
60 &11 &5 &5 &$60=$&$11(5) + 5$\\
11 &5 &2 &1 &$11=$&$5(2) + 1$\\
5 &1 &5 &0 &$5=$&$5(1) + 0$\\
\end{tabular}
\begin{align*}
1 &= 11 - 2(5)\\
&= 11 - 2(60 -5(11))\\
&= 11(11) - 2(60)\\
\end{align*}
so, $d = 11$ , check:\\
$11 * 11 \equiv 1 \pmod{60}$\\
$\hspace{.5cm}121 \equiv 1 \pmod{60}$\\
\newpage
\item [(c)]\hspace{1cm}\\
Decrypt $M \equiv C^d \pmod{77}$\\
let $a=C=32$, $n=d=11$, $m=77$\\
$n=1011$\\
\begin{align*}
r_0 &= a \pmod {77}\\
&= 32 \pmod {77}\\
\hspace{1cm}\\
r_1 &= (r_0)^2 \pmod {77}\\
&= 32^2 \pmod {77}\\
&= 23 \pmod {77}\\
\hspace{1cm}\\
r_2 &= (r_1)^2 * 32 \pmod {77}\\
&= 23^2 * 32 \pmod {77}\\
&= 65 \pmod {77}\\
\hspace{1cm}\\
r_3 &= (r_2)^2 * 32 \pmod {77}\\
&= 65^2 * 32 \pmod {77}\\
&= 65 \pmod{77}\\ 
\end{align*}
so, $M = 65$\\
%\begin{problem}
%\begin{answer}
%[Answer]
%\end{answer}
%\end{problem}
\end{problemlist}
\end{flushleft}
\end{document}